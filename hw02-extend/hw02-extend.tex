\documentclass[a4paper]{CINTA}
\title{CINTA HW02-extend}
\author{hhh937meow}
\begin{document}
\maketitle
\begin{center}
    年级:\underline{2023\hspace{1.5cm}} 
    姓名:\underline{黄海桦\hspace{1.5cm}} 
    学号:\underline{20232131017\hspace{1.5cm}} 
\end{center}

\textbf{问题 1.} 对任意给定的一个素数$p$,求出$\mathbb{Z}_p^*$的最小生成元。任取给定一个整数$n$,对所有大于等于$1$小于$n$的素数$p$,求$\mathbb{Z}_p^*$的最小生成元。令$n = 1000000$,给出以上最小生成元集合中的最大值,及其所对应的素数$p$。

\textbf{思路:}
按照cinta中给出的思路,我们首先要解决的就是\text{find\_all\_prime\_factors}的问题。\\
那么我们可以使用贪心的思想,对于一个数$n$,我们一直用$2$整除它,直到不能整除为止,然后我们就可以得到一个数$m$,使得$n = 2^k \cdot m$,接下来我们就可以用$m$去除以奇数(从$3$到$sqrt(n)$)。这样,我们就找到了$n$的全部素因子。\\

在其次,对于最后生成$\mathbb{Z}_p^*$的$p$时,要生产大量的素数,查了一下网上,了解到了埃氏筛法。\\
那么就用埃氏筛法生成$1$到$n$的所有素数。\\

剩下的就是比较公式化的py操作了。\\

\textbf{代码:}
\begin{minted}[frame=single, fontsize=\small]{python}
    import math

def sieve_primes(limit):
    # 埃氏筛法
    is_prime = [True] * limit
    is_prime[0:2] = [False, False]
    for i in range(2, int(math.sqrt(limit)) + 1):
        if is_prime[i]:
            for j in range(i * i, limit, i):
                is_prime[j] = False
    return [i for i, val in enumerate(is_prime) if val]

def get_prime_factors(n):
    factors = set()
    # 先除以2
    while n % 2 == 0:
        factors.add(2)
        n //= 2
    # 再试奇数因子
    for i in range(3, int(math.sqrt(n)) + 1, 2):
        while n % i == 0:
            factors.add(i)
            n //= i
    # 最后剩下的是一个大质因子
    if n > 2:
        factors.add(n)
    return list(factors)

def find_min_generator(p):
    if p <= 2:
        return None
    factors = get_prime_factors(p - 1)
    for g in range(2, p):
        if all(pow(g, (p - 1) // q, p) != 1 for q in factors):
            return g
    return None

def find_all_min_generators(n):
    result = {}
    primes = sieve_primes(n)
    for p in primes:
        if p > 2:
            g = find_min_generator(p)
            if g:
                result[p] = g
    return result

def find_max_generator_from_dict(gen_dict):
    max_value = max(gen_dict.values())
    max_keys = [p for p, g in gen_dict.items() if g == max_value]
    return max_keys, max_value

if __name__ == "__main__":
    n = int(input("Enter a number: "))
    generator_dict = find_all_min_generators(n)
    max_keys, max_value = find_max_generator_from_dict(generator_dict)
    print(f"\n最大生成元值为 {max_value},对应的素数 p 为: {max_keys}")
\end{minted}

\textbf{运行结果:}
Enter a number: 1000000 \\
最大生成元值为 73,对应的素数 p 为: [760321]
\end{document}