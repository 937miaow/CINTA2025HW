\documentclass[a4paper]{CINTA}
\title{CINTA HW01}
\author{hhh937meow}
\begin{document}
\maketitle
\begin{center}
    年级:\underline{2023\hspace{1.5cm}} 
    姓名:\underline{黄海桦\hspace{1.5cm}} 
    学号:\underline{20232131017\hspace{1.5cm}} 
\end{center}

\textbf{问题 1.} 用Python实现以下算法:简单乘法、模指数运算、GCD和EGCD。都必须是迭代算法。\\

\texttt{simple\_multiply:}
\begin{minted}[frame=single, fontsize=\small]{python}
  def is_even(n):
      return n & 1 == 0

  def simple_multiply_iter(a, b):
    result = 0
    while b != 0:
      if is_even(b):
        a, b = 2*a, b >> 1
      else:
        result += a
        a, b = 2*a, b >> 1
    return result
\end{minted}

\texttt{mod\_exp:}
\begin{minted}[frame=single, fontsize=\small]{python}
  def mod_exp_iter(x, y, p): #x^y mod p
    result = 1
    x = x % p
    while y > 0:
        if y & 1:
            result = (result * x) % p
        y = y >> 1
        x = (x * x) % p
    return result
\end{minted}

\texttt{gcd:}
\begin{minted}[frame=single, fontsize=\small]{python}
  def gcd_iter(a,b):
    while b != 0:
        a, b = b, a % b
    return a
\end{minted}

\texttt{egcd:}
\begin{minted}[frame=single, fontsize=\small]{python}
  def egcd_iter(a, b):
    r0, r1 = 1, 0   # |r0 s0 a|
    s0, s1 = 0, 1   # |r1 s1 b|
    while b != 0:
        c = a // b
        r0, r1 = r1, r0 - c * r1
        s0, s1 = s1, s0 - c * s1
        a, b = b, a % b
    return a, r0, s0
\end{minted}

\textbf{问题 2.} 设$n$是大于零的整数。对任意$n + 1$个正整数,其中必然存在两个数的差是$n$的倍数。\\
$\because$ 对于模$n$,其余数为0,1,2,...,$n-1$,共$n$个。\\
$\therefore$ 对于$n + 1$个数,根据抽屉原理,至少有两个数a, b会具有相同的余数 \\
即 $a \equiv b \pmod{n} \Rightarrow a - b \equiv 0 \pmod{n}$ \\
$\therefore$ 存在两个数的差是$n$的倍数。\\

\textbf{问题 3.} 计算$(2^{70} + 3^{70}) mod 13$。\\
$\because 2^6 \equiv -1 \pmod{13} 且 2^{70} = 2^{6*11+4},2^{4} mod 13 = 3$ \\
$\therefore 2^{70} \equiv -3 \pmod{13}$ \\
又$\because 3^3 \equiv 1 \pmod{13} 且 3^{70} = 2^{3*23+1},3^{1} mod 13 = 3$ \\
$\therefore 3^{70} \equiv 3 \pmod{13}$ \\
$\therefore (2^{70} + 3^{70}) \equiv 0 \pmod{13}$ \\
$\therefore (2^{70} + 3^{70}) mod 13 = 0$\\

\textbf{问题 4.} 使用费尔马小定理求解同余方程$x^{51} \equiv 2 \pmod{17}$。\\
由费马小定理可知:$a^{16} \equiv 1 \pmod{17}$ \\
则有:$x^{16n+1} \equiv x \pmod{17}$ \\
又$\because$ $gcd(16,51) = 1$
$\therefore$ 令 $51k = 16n+1, n = 0,1,2,...$ \\
解得:$k=11$ \\
$\because$有:$x^{51*11} \equiv x \equiv 2^{11} \pmod{17}$ \\
解得:$x \equiv 8 \pmod{17}$ \\
$\therefore$ x = 8\\

\textbf{问题 5.} 求解一元同余方程$6x \equiv 9 \pmod{23}$ \\
$\because$ $gcd(6,23) = 1$ \\
$\therefore$ 有解 \\
$\because$ $6*4 \equiv 1 \pmod{23}$ \\
$\therefore$ 两边同时乘4,得:$x \equiv 4*9 \equiv 13 \pmod{23}$ \\
$\therefore$ x = 13\\ 
\end{document}