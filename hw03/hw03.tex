\documentclass[a4paper]{CINTA}
\title{CINTA HW03}
\author{hhh937meow}
\begin{document}
\maketitle
\begin{center}
    年级:\underline{2023\hspace{1.5cm}} 
    姓名:\underline{黄海桦\hspace{1.5cm}} 
    学号:\underline{20232131017\hspace{1.5cm}} 
\end{center}

\textbf{问题 1.}给定任意群$\mathbb{G}$,$\mathbb{H}$是群$\mathbb{G}$的正规子群。请证明,如果群$\mathbb{G}$是阿贝尔群,则商群$\mathbb{G}/\mathbb{H}$也是阿贝尔群。\\
\textbf{证明:} \\
要证明$\mathbb{G}/\mathbb{H}$是阿贝尔群,我们需要证明对于任意两个元素$a\mathbb{H},b\mathbb{H} \in \mathbb{G}/\mathbb{H}$,都有$(a\mathbb{H})(b\mathbb{H})=(b\mathbb{H})(a\mathbb{H})$ \\
根据商群的运算定义,有:$(a\mathbb{H})(b\mathbb{H})=(ab)\mathbb{H}$ \\
同时,$(b\mathbb{H})(a\mathbb{H})=(ba)\mathbb{H}$ \\
$\because \mathbb{G}$是阿贝尔群,所以$ab=ba$,因此$(ab)\mathbb{H}=(ba)\mathbb{H}$,即$(a\mathbb{H})(b\mathbb{H})=(b\mathbb{H})(a\mathbb{H})$ \\
$\therefore$商群$\mathbb{G}/\mathbb{H}$也是阿贝尔群 \\

\textbf{问题 2.}给定任意群$\mathbb{G}$,$\mathbb{H}$是群$\mathbb{G}$的正规子群。请证明,如果群$\mathbb{G}$是循环群,则商群$\mathbb{G}/\mathbb{H}$也是循环群。
\textbf{证明:} \\
已知群$\mathbb{G}$是循环群,所以存在一个生成元$g \in \mathbb{G}$,使得$\mathbb{G}=<g>$ \\
则群$\mathbb{G}$的任意元素都可以表示为$g^n$,其中$n \in \mathbb{Z}$ \\
要证明商群$\mathbb{G}/\mathbb{H}$是循环群,我们需要找到一个元素$g\mathbb{H} \in \mathbb{G}/\mathbb{H}$,使得$\mathbb{G}/\mathbb{H}=<g\mathbb{H}>$ \\
取$g\mathbb{H}$为商群$\mathbb{G}/\mathbb{H}$的一个元素,考虑$(g\mathbb{H})^n$,其中$n \in \mathbb{Z}$ \\
根据商群的运算定义,有:$(g\mathbb{H})^n=g^n\mathbb{H}$ \\
不难得知,对所有的正整数$n$都成立 \\
现在考虑负整数的情况,设$n=-m$,其中$m \in \mathbb{Z}^+$,则有:\\
$(g\mathbb{H})^{-m}=((g\mathbb{H})^{-1})^m$ \\
$\because (g\mathbb{H})(g^{-1}\mathbb{H})=(gg^{-1}\mathbb{H})=e\mathbb{H}=\mathbb{H}$ \\
且 $(g^{-1}\mathbb{H})(g\mathbb{H})=(g^{-1}g\mathbb{H})=e\mathbb{H}=\mathbb{H}$ \\
$\therefore$$(g\mathbb{H})^{-m}=(g^{-1}\mathbb{H})^m=(g^{-1})^m\mathbb{H}=g^{-m}\mathbb{H}$ \\
$\therefore$,商群$\mathbb{G}/\mathbb{H}$中的任意元素都可以表示为$(g\mathbb{H})^n=g^n\mathbb{H}$,其中$n \in \mathbb{Z}$ \\
$\therefore$,商群$\mathbb{G}/\mathbb{H}$是循环群 \\

\textbf{问题 3.}设$p$和$q$是两个不同的素数,$\mathbb{G}$是阶为$pq$的阿贝尔群,请证明$\mathbb{G}$是循环群。\\
\textbf{证明:} \\
要证明$\mathbb{G}$是循环群,我们需要证明$\mathbb{G}$中存在一个元素的阶为$pq$ \\
根据拉格朗日定理,群$\mathbb{G}$的阶为$pq$,则$\mathbb{G}$中任意元素的阶只能是$p$、$q$、$pq$ \\
设$\mathbb{G}$中存在一个元素$g$ \\
若$g$的阶为$pq$,则$\mathbb{G}=<g>$,所以$\mathbb{G}$是循环群 \\
下面假设$\mathbb{G}$中不存在一个阶为$pq$的元素 \\
那么$\mathbb{G}$中所有元素的阶只能是$p$或$q$ \\
如果$\mathbb{G}$中所有非单位元的元素的阶都是$p$,根据群的性质,这要求群的阶是$p$的幂次。但是$|\mathbb{G}|=pq$,这不是$p$的幂次 \\
同理,如果$\mathbb{G}$中所有非单位元的元素的阶都是$q$,根据群的性质,这要求群的阶是$q$的幂次。但是$|\mathbb{G}|=pq$,这不是$q$的幂次 \\
所以$\mathbb{G}$中至少存在一个阶为$p$的元素和一个阶为$q$的元素 \\
设$a$的阶为$p$,$b$的阶为$q$ \\
又$\because \mathbb{G}$是阿贝尔群,则$ab=ba$ \\
$\because (ab)^{pq}=a^{pq}b^{pq}=e^qe^p=e$ \\
$\therefore$ $ab$的阶整除$pq$。设$|ab|=k$,则$k \in \{1,p,q,pq\}$ \\
如果$k=1$,则$ab=e$,这与$a$和$b$的阶矛盾 \\
如果$k=p$,则$(ab)^p=a^pb^p=e^qe^p=e$,这与$b$的阶矛盾 \\
如果$k=q$,则$(ab)^q=a^qb^q=e^qe^p=e$,这与$a$的阶矛盾 \\
如果$k=pq$,则$(ab)^{pq}=a^{pq}b^{pq}=e^qe^p=e$ \\
所以$|ab|=pq$,即$ab$的阶为$pq$ \\
$\therefore$,$\mathbb{G}$是循环群 \\

\textbf{问题 4.}证明:$\mathbb{GL}_n(R)/\mathbb{SL}_n(R)$与乘法群$\mathbb{R}^*$同构。\\
\textbf{证明:} \\
设$\mathbb{GL}_n(R)$是$n$阶可逆矩阵的集合,$\mathbb{SL}_n(R)$是行列式为1的$n$阶可逆矩阵的集合 \\
考虑映射$\phi:\mathbb{GL}_n(R) \to \mathbb{R}^*$,定义为$\phi(A)=\det(A)$,其中$A \in \mathbb{GL}_n(R)$ \\
$\because$对于任意的$A,B \in \mathbb{GL}_n(R)$,都有:$\phi(AB)=\det(AB)=\det(A)\det(B)=\phi(A)\phi(B)$ \\
$\therefore$$\phi$是一个同态映射 \\
对于任意$r \in \mathbb{R}^*$,存在$A \in \mathbb{GL}_n(R)$,使得$\det(A)=r$ \\
$\therefore$,$\phi$是满射,$Im(\phi)=\mathbb{R}^*$ \\
易知$Ker(\phi)=\mathbb{SL}_n(R)$ \\
$\therefore$,$\mathbb{GL}_n(R)/\mathbb{SL}_n(R) \cong \mathbb{R}^*$ \\

\textbf{问题 5.}设$\phi: \mathbb{G}\mapsto \mathbb{H}$是一种群同态。请证明$\phi$是单射当且仅当$\mbox{Ker}\; \phi  = \{e\}$。\\
\textbf{证明:} \\
$\Rightarrow$:假设$\phi$是单射,$\therefore$对于任意的$x,y \in \mathbb{G}$,如果$\phi(x)=\phi(y)$,则$x=y$ \\
设$\phi(x)=\phi(e)$,则$x=e$ \\
$\therefore$$\phi$的核只有单位元,所以$Ker(\phi)= \{e\}$ \\

$\Leftarrow$:假设$Ker(\phi)  = \{e\}$,$\therefore$对于任意的$x,y \in \mathbb{G}$,如果$\phi(x)=\phi(y)$,则$\phi(xy^{-1})=\phi(x)\phi(y^{-1})=\phi(e)=e$ \\
$\therefore$,$xy^{-1} \in Ker(\phi)$ \\
$\therefore$,$xy^{-1}=e$ \\
$\therefore$,$x=y$ \\
$\therefore$,$\phi$是单射 \\
综上所述,$\phi$是单射当且仅当$Ker(\phi)= \{e\}$ \\

\textbf{问题 6.}给定有限阿贝尔群$\mathbb{G}$和正整数$n$,定义映射$\phi: \mathbb{G}\mapsto \mathbb{G}$为,$\forall g \in \mathbb{G}$,$\phi(g) = g^n$。
请分析并证明,在什么条件下$\phi$会是一种单射?[提示:利用以上证明的结论。]\\
\textbf{证明:} \\
对于有限阿贝尔群$\mathbb{G}$和映射$\phi(g) = g^n$,$\forall g,h \in \mathbb{G}$,都有$\phi(gh)=(gh)^n=g^nh^n=\phi(g)\phi(h)$ \\
$\therefore$,$\phi$是一个同态映射 \\
由问题五的结论可得,一个群同态是单射当且仅当它的核只包含定义域的单位元 \\
即:$\mbox{Ker}\; \phi = \{g \in \mathbb{G} \mid g^n = e_{\mathbb{G}}\}$ \\
也就是除了单位元之外,没有其他元素的$n$次幂为单位元 \\
即:除了单位元之外,没有其他元素的阶整除$n$ \\
所以$\phi$是单射当且仅当$gcd(|\mathbb{G}|,n)=1$ \\

$\Rightarrow$:
假设$\phi$是单射,$\therefore$对于任意的$x,y \in \mathbb{G}$,如果$\phi(x)=\phi(y)$,则$x=y$ \\
设$\phi(x)=\phi(e)$,则$x=e$ \\
$\therefore$$\phi$的核只有单位元,所以$Ker(\phi)= \{e\}$ \\
$\therefore$$gcd(|\mathbb{G}|,n)=1$ \\

$\Leftarrow$:
假设$gcd(|\mathbb{G}|,n)=1$,$\therefore$对于任意的$x,y \in \mathbb{G}$,如果$\phi(x)=\phi(y)$,则$\phi(xy^{-1})=\phi(x)\phi(y^{-1})=\phi(e)=e$ \\
$\therefore$$xy^{-1} \in Ker(\phi)$
$\therefore$$xy^{-1}=e$ \\
$\therefore$$x=y$ \\
$\therefore$$\phi$是单射 \\
综上所述,$\phi$是单射当且仅当$gcd(|\mathbb{G}|,n)=1$ \\

\end{document}