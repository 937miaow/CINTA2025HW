\documentclass[a4paper]{CINTA}
\title{CINTA HW05}
\author{hhh937meow}
\begin{document}
\maketitle
\begin{center}
    年级:\underline{2023\hspace{1.5cm}} 
    姓名:\underline{黄海桦\hspace{1.5cm}} 
    学号:\underline{20232131017\hspace{1.5cm}} 
\end{center}

\textbf{问题 1.}请计算以下勒让德符号:$(\frac{7}{11})$,$(\frac{72}{131})$,$(\frac{20}{23})$,$(\frac{17}{31})$。\\
\textbf{解:}\\
由欧拉准则可知:$\left(\frac{a}{p}\right) \equiv a^{(p-1)/2} \mod p$\\
\textbf{(1)} $(\frac{7}{11})$\\
\begin{align*}
    7^{(11-1)/2} &\equiv 7^5 \mod 11\\
    &\equiv 7^4 \cdot 7 \mod 11\\
    &\equiv 3 \cdot 7 \mod 11\\
    &\equiv 10 \mod 11\\
    &\equiv -1 \mod 11\\
    &\Rightarrow (\frac{7}{11}) = -1
\end{align*}

\textbf{(2)} $(\frac{72}{131})$\\
\begin{align*}
    72^{(131-1)/2} &\equiv 72^{65} \mod 131\\
    &\equiv 72^{64} \cdot 72 \mod 131\\
    &\equiv 20 \cdot 72 \mod 131\\
    &\equiv -1 \mod 131\\
    &\Rightarrow (\frac{72}{131}) = -1
\end{align*}

\textbf{(3)} $(\frac{20}{23})$\\
\begin{align*}
    20^{(23-1)/2} &\equiv 20^{11} \mod 23\\
    &\equiv 20^{10} \cdot 20 \mod 23\\
    &\equiv 8 \cdot 20 \mod 23\\
    &\equiv -1 \mod 23\\
    &\Rightarrow (\frac{20}{23}) = -1
\end{align*}

\textbf{(4)} $(\frac{17}{31})$\\
\begin{align*}
    17^{(31-1)/2} &\equiv 17^{15} \mod 31\\
    &\equiv 17^{14} \cdot 17 \mod 31\\
    &\equiv 20 \cdot 17 \mod 31\\
    &\equiv -1 \mod 31\\
    &\Rightarrow (\frac{17}{31}) = -1
\end{align*}

\textbf{问题 2.}设$n= 7*11$,请手动计算求出同余方程$x^2 \equiv a \pmod{n}$在$a = 5$和$a = 4$时的所有解。\\
\textbf{解:}\\
由中国剩余定理可得:\\
\begin{align*}
    x^2 &\equiv a \mod 7\\
    x^2 &\equiv a \mod 11
\end{align*}

\textbf{(1)} $a = 5$\\
$\because$ 模7的二次剩余有$0,1,2,4$,模11的二次剩余有$0,1,3,4,5,9$\\
对于模7的二次剩余没有5,所以$x^2 \equiv 5 \mod 7$没有解\\
则对于原方程$x^2 \equiv 5 \mod 77$也没有解\\

\textbf{(2)} $a = 4$\\
由(1)可知:$a = 4$有解\\
\begin{align*}
    x^2 &\equiv 4 \mod 7\\
    x^2 &\equiv 4 \mod 11
\end{align*}
对于$x^2 \equiv 4 \mod 7$,可分解为:\\
\begin{align*}
    (x-2)(x+2) &\equiv 0 \mod 7\\
    x &\equiv 2 \mod 7\\
    x &\equiv -2 \mod 7\\
    x &\equiv 5 \mod 7
\end{align*}
对于$x^2 \equiv 4 \mod 11$,可分解为:\\
\begin{align*}
    (x-2)(x+2) &\equiv 0 \mod 11\\
    x &\equiv 2 \mod 11\\
    x &\equiv -2 \mod 11\\
    x &\equiv 9 \mod 11
\end{align*}
现在我们需要使用中国剩余定理组合这些解。共有 $2 * 2=4$ 组可能的解。\\
\textbf{(1)}\\
\begin{align*}
    x &\equiv 2 \mod 7\\
    x &\equiv 2 \mod 11
\end{align*}
观察易知,$x \equiv 2 \mod 77$\\

\textbf{(2)}\\
\begin{align*}
    x &\equiv 2 \mod 7\\
    x &\equiv 9 \mod 11
\end{align*}
由中国剩余定理,$x = 2 * 11 * 11^{-1} + 9 * 7 * 7^{-1} = 2 * 11 * 2 + 9 * 7 * 8 = 9 \mod{77}$

\textbf{(3)}\\
\begin{align*}
    x &\equiv 5 \mod 7\\
    x &\equiv 2 \mod 11
\end{align*}
由中国剩余定理,$x = 5 * 11 * 11^{-1} + 2 * 7 * 7^{-1} = 5 * 11 * 2 + 2 * 7 * 8 = 68 \mod{77}$

\textbf{(4)}\\
\begin{align*}
    x &\equiv 5 \mod 7\\
    x &\equiv 9 \mod 11
\end{align*}
由中国剩余定理,$x = 5 * 11 * 11^{-1} + 9 * 7 * 7^{-1} = 5 * 11 * 2 + 9 * 7 * 8 = 75 \mod{77}$

\textbf{总结:}\\
$x^2 \equiv 4 \mod 77$的解为:\\
\begin{align*}
    x &\equiv 2 \mod 77\\
    x &\equiv 9 \mod 77\\
    x &\equiv 68 \mod 77\\
    x &\equiv 75 \mod 77
\end{align*}

\end{document}